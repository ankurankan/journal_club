\documentclass{beamer}

\usepackage[utf8]{inputenc}
\usecolortheme{beaver}
\usepackage{caption}
\usepackage{amsmath}
\usepackage{amsfonts}
\usepackage{amssymb}
\usepackage{subcaption}
\usepackage{mathtools}
\usepackage{bm}
\usepackage[style=verbose, backend=biber]{biblatex}

\begin{document}

% \title{Discussion: Partially Adaptive Regularized Multiple Regression Analysis for Estimating Linear Causal Effects}

% \begin{frame}
% 	\frametitle{Summary}
% 	\begin{itemize}
% 		\item Deals with the case when multi-colinearirty is present in the regression equation.
% 		\item Many ways to deal with it in the non-causal case but they don't work well for causal estimation task.
% 		\item Normal methods may remove some of the covariates from the regression equation depending on it's parameter value,
% 			but for causal estimation it is important to keep them in the equation.
% 		\item Collapsibility condition can be derived to show that the proposed method is consistent with OLS.
% 	\end{itemize}
% \end{frame}
% 
% \begin{frame}
% 	\frametitle{Positive points}
% \end{frame}
% 
% \begin{frame}
% 	\frametitle{Negative Points}
% 	\begin{itemize}
% 		\item More intuition for why it's working.
% 		\item Further compare other methods for selecting variables.
% 	\end{itemize}
% \end{frame}

\begin{frame}
	\frametitle{Discussion Points}
	\begin{itemize}
		\item Best strategies for selecting the confounding set, $
			\bm{W} $ and the set of possible confounders $ \bm{Z}
			$. Can the choice affect the estimates?
		\item How should we choose the optimal hyper-parameters? In the
			paper a cross-validation approach is used but that
			could be baised toward predictibility instead of true
			causal estimates. Maybe some simulation method can be
			used?
		\item Challenges in extending it to other statistical models
			like GLM, Generalized estimating equations, and
			proportional hazards model.
	\end{itemize}
\end{frame}

\end{document}
