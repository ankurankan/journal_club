\documentclass{beamer}

\usepackage[utf8]{inputenc}
\usecolortheme{beaver}
\usepackage{caption}
\usepackage{subcaption}
\usepackage{mathtools}
\usepackage{todonotes}
\usepackage{amsmath}
\usepackage{bm}
\usepackage{listings}
\usepackage{ragged2e}
\usepackage{titlecaps}
\usepackage{fancyvrb}

\def\ci{\perp\!\!\!\!\!\perp}

\newtheorem{proposition}{Proposition}
\Addlcwords{for a is but and with of in as the etc on to if}

\setbeamertemplate{section in toc}{\inserttocsectionnumber.~\inserttocsection}
\usetheme{Boadilla}
\makeatletter
\setbeamertemplate{footline}{%
    \leavevmode%
    \hbox{%
        \begin{beamercolorbox}[wd=.3\paperwidth,ht=2.25ex,dp=1ex,center]{author in head/foot}%
            \usebeamerfont{author in head/foot}\insertshortauthor\expandafter\beamer@ifempty\expandafter{\beamer@shortinstitute}{}{~~(\insertshortinstitute)}
        \end{beamercolorbox}%
        \begin{beamercolorbox}[wd=.55\paperwidth,ht=2.25ex,dp=1ex,center]{title in head/foot}%
            \usebeamerfont{title in head/foot}\insertshorttitle
        \end{beamercolorbox}%
        \begin{beamercolorbox}[wd=.15\paperwidth,ht=2.25ex,dp=1ex,right]{date in head/foot}%
            \usebeamerfont{date in head/foot}\insertshortdate{}\hspace*{2em}
            \insertframenumber{} / \inserttotalframenumber\hspace*{2ex} 
        \end{beamercolorbox}}%
        \vskip0pt%
    }
\makeatother

\begin{document}

\title[]{Residualization Based Conditional Independence Test for Mixed Data}
\author{Ankur Ankan}
\date{}

\maketitle

\begin{frame}{Structural Equation Models}
	Give an example of a SEM.
\end{frame}

\begin{frame}{Parameterization of a Linear SEM}
	Give an example of another SEM.
	Give some meaning on iterpreting these parameters.
	Give example of path analysis.
\end{frame}

\begin{frame}{Estimating these parameters}
	Can use linear regression.
	Conditioning on the covariates.
\end{frame}

\begin{frame}{Estimating these through residauls}
	Introduce the FWL theorem.
	What applying this to estimation of looks like.
\end{frame}

\begin{frame}{What happens in the case of mixed data}
	What we require to extend this estimation to mixed data.
	1. Some way to have residuals for mixed data.
	2. Some kind of effect size measure for this mixed residuals.
\end{frame}

\begin{frame}{Residuals: Continuous Variablese}
	Simple difference between true and predicted values.
\end{frame}

\begin{frame}{Residuals: Ordinal Variables}
	LS residuals. With example
\end{frame}

\begin{frame}{Residuals: Categorical variables}
	Categorical variables with examples.
\end{frame}

\begin{frame}{Residuals Summary}
	We get a matrix of continuous valued residuals for each case.
\end{frame}

\begin{frame}{Effect Size: Canonical Correlation}
	Can't directly use correlation coefficient as we have matrix of
	residuals but canonical correlation based things can be used.

	What is canonical correlation?
\end{frame}

\begin{frame}
	How does these parameters look like. Show on the adult income dataset
	or something.

	This also has the multiplicative property of path analysis. Show an
	example.
\end{frame}

\begin{frame}{CI Testing based on canonical correlation}
	Can use something like Pillai's trace.
\end{frame}

\begin{frame}{Results of CI testing}
	Canonical correlation results in better calibration and power in high
	dimensional settings.
\end{frame}

\begin{frame}{Conclusion/Remaining problems}
	Not sure how addition rule of path coefficients would work.
	For CI tests, asymptotic theory isn't clear.
\end{frame}

\end{document}
