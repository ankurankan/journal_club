\documentclass{beamer}

\usepackage[utf8]{inputenc}
\usecolortheme{beaver}
\usepackage{caption}
\usepackage{subcaption}
\usepackage{mathtools}
\usepackage{todonotes}
\usepackage{amsmath}
\usepackage{bm}
\usepackage{listings}
\usepackage{ragged2e}
\usepackage{titlecaps}
\usepackage{fancyvrb}

\def\ci{\perp\!\!\!\!\!\perp}

\newtheorem{proposition}{Proposition}
\Addlcwords{for a is but and with of in as the etc on to if}

\setbeamertemplate{section in toc}{\inserttocsectionnumber.~\inserttocsection}
\usetheme{Boadilla}
\makeatletter
\setbeamertemplate{footline}{%
    \leavevmode%
    \hbox{%
        \begin{beamercolorbox}[wd=.3\paperwidth,ht=2.25ex,dp=1ex,center]{author in head/foot}%
            \usebeamerfont{author in head/foot}\insertshortauthor\expandafter\beamer@ifempty\expandafter{\beamer@shortinstitute}{}{~~(\insertshortinstitute)}
        \end{beamercolorbox}%
        \begin{beamercolorbox}[wd=.55\paperwidth,ht=2.25ex,dp=1ex,center]{title in head/foot}%
            \usebeamerfont{title in head/foot}\insertshorttitle
        \end{beamercolorbox}%
        \begin{beamercolorbox}[wd=.15\paperwidth,ht=2.25ex,dp=1ex,right]{date in head/foot}%
            \usebeamerfont{date in head/foot}\insertshortdate{}\hspace*{2em}
            \insertframenumber{} / \inserttotalframenumber\hspace*{2ex} 
        \end{beamercolorbox}}%
        \vskip0pt%
    }
\makeatother

\begin{document}

\title[]{Human in the loop Structure Learning}
\date{}

\maketitle

\begin{frame}{Background}
	A very basic example on the DAGs.
	What the nodes and edges mean.
\end{frame}

\begin{frame}{Learning DAGs from data}
	Learning DAGs from data. The general process and the main approaches.

	Plenty of algorithms to automatically learn the model.

	List some. And plenty of variations.
\end{frame}

\begin{frame}{In practice}
	However in practice, researchers do not use them and manually draw the DAGs.
	Even when we use them, the output is not perfect and requires some modification 
		before can be used.
\end{frame}

\begin{frame}{Model testing}
	Some researchers do use model testing appraoches to the built model.

	How does model testing work?

	Model testing can be helpful for doing the modifications.
	The usual approach here is that users draw by hand and then test their model. <Find references from citations of the testing paper>
\end{frame}

\begin{frame}{Another way to test models}
	Match the implied correlation structure with the observed correlation in the data.
\end{frame}

\begin{frame}{Human-in-the-loop Structure Learning}
	As most of the models are being built by hand, a way to suggest them how to do it.
\end{frame}

\begin{frame}{Human-in-the-loop SL}
	Describe the method for continuous variables.
\end{frame}

\begin{frame}{An effect size for mixed data}
\end{frame}

\begin{frame}{Demo}
\end{frame}

\end{document}
