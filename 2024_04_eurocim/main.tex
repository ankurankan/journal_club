\documentclass{beamer}
\usepackage[orientation=portrait,width=30in,height=40in,scale=1.35,debug]{beamerposter}
\mode<presentation>{\usetheme{ZH}}
\usepackage{chemformula}
\usepackage[utf8]{inputenc}
\usepackage[english]{babel} % required for rendering German special characters
\usepackage{hyperref} %enable hyperlink for urls
\usepackage{ragged2e}
\usepackage[font=small, justification=justified]{caption}
\usepackage{array,booktabs,tabularx}
\usepackage{bm}
\usepackage{subcaption}

\def\ci{\perp\!\!\!\!\!\perp}

\newcolumntype{Z}{>{\centering\arraybackslash}X} % centered tabularx columns
\title{\huge A Generalized Parameterization for \\ Mixed Data Linear Structural Equation Models}
\author{Ankur Ankan, Johannes Textor}
\institute[RU]{Institute for Computing and Information Sciences \\ Radboud University, Netherlands}
\date{\today}

% edit this depending on how tall your header is. We should make this scaling automatic :-/
\newlength{\columnheight}
\setlength{\columnheight}{106cm}

\begin{document}
\begin{frame}
\begin{columns}
	\begin{column}{.33\textwidth}
		\begin{beamercolorbox}[center]{postercolumn}
			\begin{minipage}{.98\textwidth}  % tweaks the width, makes a new \textwidth
				\parbox[t][\columnheight]{\textwidth}{ % must be some better way to set the the height, width and textwidth simultaneously
	\begin{myblock}{Introduction}
	
		\begin{itemize}
			\item Path Coefficients are easy to interpret.
			\item One unit change in $ X_2 $ while keeping $ X_4 $ fixed $ \implies \beta_{23} $ change in $ X_3 $.
			\item Indirect effects can be computed easily.
		\end{itemize}

		However, when we have mixed variables it becomes difficult. Using regression methods give us a matrix to parameters that 
		are much harder to interpret.

	\end{myblock}\vfill
	\begin{myblock}{Research Question}
		Can we define a path coefficient for mixed data SEMs?
	\end{myblock}\vfill
	\begin{myblock}{Path Coefficients and Effect Sizes}
	\end{myblock}\vfill
		}\end{minipage}\end{beamercolorbox}
	\end{column}

%%%%%%%%%%%%%%%%%%%%%%%%%%%%%%%%%%%%%%%%%%% Second Column %%%%%%%%%%%%%%%%%%%%%%%%%%%%%%%%%%%%%%%%%%%%%%%%%%%%%%%

	\begin{column}{.33\textwidth}
		\begin{beamercolorbox}[center]{postercolumn}
			\begin{minipage}{.98\textwidth} % tweaks the width, makes a new \textwidth
				\parbox[t][\columnheight]{\textwidth}{ % must be some better way to set the the height, width and textwidth simultaneously
	\begin{myblock}{Effect size for mixed data}

	\end{myblock}\vfill
	\begin{myblock}{Example}
	\end{myblock}
		}\end{minipage}\end{beamercolorbox}
	\end{column}


%%%%%%%%%%%%%%%%%%%%%%%%%% Third column %%%%%%%%%%%%%%%%%%%%%%%%%%%%%%%%%%%%%%%%%%%%%%%%%%%%%
	\begin{column}{0.33\textwidth}
		\begin{beamercolorbox}[center]{postercolumn}
			\begin{minipage}{.98\textwidth} % tweaks the width, makes a new \textwidth
				\parbox[t][\columnheight]{\textwidth}{ % must be some better way to set the the height, width and textwidth simultaneously
	\begin{myblock}{Research Question}
		Can there be a independence test based on this effect size?
	\end{myblock}\vfill
	\begin{myblock}{Pillai's Trace}
	\end{myblock}\vfill
	\begin{myblock}{Conclusion and Future Work}
		\begin{itemize}
			\item Shows the connection between path coefficients and effect sizes.
			\item Gives an effect size based on Canonical correlations for mixed data.
			\item Future work: How to interpret these coefficients.
			\item Future work: More analysis on the CI test based on this.
		\end{itemize}
	\end{myblock}\vfill
	\begin{myblock}{References}
		\footnotesize
		\bibliographystyle{abbrv}
		\bibliography{./bib}
	\end{myblock}\vfill
		}\end{minipage}\end{beamercolorbox}
	\end{column}
\end{columns}
\end{frame}
\end{document}
